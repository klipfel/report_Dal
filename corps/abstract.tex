
\vspace{50pt}

\section*{Ackowledgement}\label{thk}

This internship was a very enriching experience in my life.
Undoubtedly, the concepts and tools I have come to learn will 
guide me in my future research career. I want to deeply thank
Dr. Trappenberg for his astute and lasting support
in my work.


\clearpage

\section*{Abstract}

% VERIF : *
\introductionLettrine{T}{he} concept of tracking robot can bring trailblazing and 
game-changing applications either in the industry, defense or
at home. More practically, robots could be seen as
the future coworkers of humans. In that context, 
robots will have at some point to be able to 
not only follow instructions, but assist for instance
workers in a factory, and so be able to 
follow their track. One can imagine a robot that could 
carry tools, follow any worker, and hand out the right one 
when asked.
\\\indent Few trackers following general targets, not just 
human ones, have been realized using deep learning
techniques, and even fewer using the middleware
\textit{ROS}. As a matter of fact, this project aimed to 
realize a versatile \textit{ROS} architecture
for a tracking robot using a stereo camera
and deep learning algorithms to spot a 
general given target.
\\\indent The project was divided into 
several main stakes. First of all, the hardware
of the tracker had to be wisely selected to 
meet the requirements of the tracking process. 
Secondly, the tracking algorithm had to be chosen 
and assessed. The \textit{ROS} architecture had 
then to be imagined and built. Finally, the thorniest 
part was to integrate each part on the robot.
\\\indent At the end, a flexible and general \textit{ROS}
architecture was implemented and tested in diverse 
conditions. The realized architecture 
could then inspire future work, and be the basis 
of some improvements thanks to the flexibility of \textit{ROS}.


\clearpage
\section*{Introduction}
% VERIF : *
\introductionLettrine{T}{his} document is the report of the project on which 
I worked during my summer internship in 2019 at Dalhousie 
University. Additional codes and documentations
could be found on my personal \href{https://github.com/klipfel}{\textit{github}}.
The main goal was to build a tracking robot
with the middleware ROS and with deep learning techniques.
\\\indent The part \vref{context} contextualize the project, 
presents the goals and the strategy. The part \vref{stateofart}
outlines several projects or existing techniques that 
were relevant for the project in any way. The part 
\vref{hardware} explains how the hardware of the 
tracker was selected. The part \vref{software}
justifies the choices made regarding the 
tracking algorithm and the \textit{ROS} architecture.
The part \vref{results} analyzes the results of the integration
of each single part on the robot, that is to say the hardware and 
the software. Finally, the part \vref{setup} gives the procedure 
and some pieces of advice in order to replicate the latest blueprint
of the tracker.

%\section*{Abstract}
\section*{Keywords} stereo-vision, tracking, \textit{ROS},
mechatronics, mobile robotics, RC car, ground robot, deep learning.

%\selectlanguage{french}
%\section*{Mots clés}


%%% Local Variables: 
%%% mode: latex
%%% TeX-master: "../guide"
%%% End: 
