\clearpage
\newgeometry{hmargin=2.5cm,vmargin=2cm}

\chapter{Project Contextualization}\label{context}

	\introductionLettrine{T}{his} part gives a first and general glimpse into what was at stake in 
	the realization of the tracking robot. The relevance of the 
	project is justified, the requirements and goals of the project
	are defined, and the global strategy of development is
	presented.

	\section{Project Specification}
	
		\subsection{Motivations}
	
		The main goal of this project was to build a \textit{Deep Learning Tracking Robot}. Many tracking
		robots have been implemented in the previous years, few have used the adaptive, versatile, and 
		now standard robotic Middleware \textit{ROS}\footnote{Which stands for \textit{Robot Operating System}.}.
		In addition, even fewer have ventured combining \textit{ROS} and \textit{Deep Learning} techniques.
		\\\indent Why \textit{ROS} precisely? As defined by Anis Kouba in \cite{ros}, \textit{ROS}
		is an \frstg{} open-source \textit{middleware} \lstg{} that provides a robust and reliable framework 
		to build robots. \textit{ROS} is a voguish tool to create robots with advanced functionalities so much so 
		that it has become the standard to develop robots. \textit{ROS2} is already in part 
		operational. In this project though, 
		it has been decided that the first version of \textit{ROS} will be used since \textit{ROS2} is still 
		in heavy development and \textit{ROS} is better documented than its counterpart. Besides, 
		some sensors used in this project, such as
		the \textit{ZED camera}\footnote{See \vref{hardwareoverview}.},
		have packages that are still in beta version, hence less reliable than the ones
		of \textit{ROS}.
		\\\indent Why is \textit{Deep Learning} suitable for robotic applications? In general, \textit{AI}
		\footnote{Artificial Intelligence, which comprises \textit{Deep Learning}.}
		is really adapted for critical algorithms, such as embedded codes
		on robots, since the performances are much higher than with traditional implementations. By traditional
		implementations, one could for instance consider iterative algorithms where for a particular task 
		all processes are hand-implemented. The conventional solution 
		that comes to mind regarding tracking is an \textit{RGB} tracking, that is to say an 
		algorithm that tracks the color in a frame\footnote{image.} by applying 
		color filters. \textit{AI} enables much finer and more robust
		implementations since the \textit{model},
		which can be regarded as the applied processes, can evolve by itself through diverse methods such as training.
		 
		\subsection{Requirements}
		
		In order to realize the tracking robot, several requirements had to be met in this
		project. First of all, the robot had to reuse an old \textit{Race Car} platform comprising
		the chassis, the wheels \dots all the mechanics. The platform is the \textit{H1 model}
		of the \textit{monster Car} \cite{datasheet}.
		The brain of the robot, that is to say 
		the embedded computer had to be the \textit{Jetson Nano} designed by \textit{Nvidia} \cite{nano}. By and
		large, to build the tracker the material of the laboratory had to be used as much as 
		possible, and the equipment bought had to remain affordable.
		%TODO image of the old platform rc car
		%TODO image of the jetson nano
	
	\section{Strategy}
	
		\subsection{Different periods}

		Developing a robot is a critical task, and has to be conducted thoroughly. 
		In this regard, \textit{ROS} builds a framework for developing 
		robots in a more organized way. For each task to implement, 
		the work-flow has always been the following: research, simulation, isolated 
		tests, integration.
		\\\indent Before even implementing something and integrating it, one should 
		research all the possibilities that are offered to them. Then one of the 
		easier or maybe most sustainable options could be chosen. Yet, even the
		tests, simulation, and integration has to be thought and foreseen beforehand.
		\\\indent Simulating robots, without depending on the hardware is
		a needed step.	 A specific hardware has always some specificities that
		could hide some issues and unpredictable behaviors. As a matter of fact, 
		simulating algorithms that will be integrated in the robot afterward in 
		a controlled environment prevents any hardware based issue to shadow
		our understanding of the basic implementation of
		the \textit{software}.
		\\\indent Once the simulation is in place, it is then 
		recommended to test the simulation as much as possible in order
		to unveil some detrimental singularities. In the case of the hardware, 
		each component has to be tested independently before integration
		\footnote{Especially, some driver issues are sometimes really 
		hard to pinpoint.}.
		\\\indent The final step is to integrate the work in the robot, and
		to test it to see if the robot behaves as expected or if there 
		is any regression.
		\\\indent A robot is not just a piece of software, it is a system 
		where the interaction of the software and the hardware is by 
		definition the future behavior of the robot.
		
		\subsection{Different aspects}

		The development of a robot encapsulates different areas or type of work.
		In the construction of the tracker, principally two aspects 
		have been tackled, the hardware, and the software.
		\\\indent In the first place, the hardware has to be well-chosen beforehand.
		The motor had to replaced, the \textit{wifi} access had to be added, batteries 
		had to be chosen \dots to fit the requirements of the tracking robot.
		\\\indent In the second place, the sofware has to be implemented. Precisely, 
		the tracking algorithm had to be chosen, and the \textit{ROS} architecture had 
		to be designed.
		\\\indent For the entire project, the philosophy was to try to have 
		a working first prototype as soon as possible. In a nutshell, 
		having a hardware and software which are compatible, and then refining
		the existing platform. Indeed, \textit{ROS} provides here another
		crucial boon, for it enables us to have an evolutive architecture 
		where each part can be replaced easily without 
		undermining the overhaul process. For instance, once 
		an architecture is working, the tracking technique could
		be easily replaced.
		\\\indent Throughout this report, which presents the 
		results of the project, three questions will be answered:
		\begin{itemize}
			\item[\textbullet] How was the hardware selected?
			You can find the answers to 
			this question in the part \vref{hardware}.
			\item[\textbullet] How was the tracking algorithm implemented? Which belongs 
			more to the software. You can find the answers to 
			this question in the section \vref{tracking}.
			\item[\textbullet] How was the \textit{ROS} architecture designed? 
			Which belongs more to the software. You can find the answers to 
			this question in the section \vref{ros}.
		\end{itemize}
		
\chapter{State of the art}\label{stateofart}

		\introductionLettrine{I}{n} order to build the tracker, the first step was to choose the
		hardware of the robot. Secondly, the \textit{ROS} architecture had to be decided
		and thirdly the tracking algorithm had to be implemented. This part 
		presents a brief overview of different existing projects and solutions 
		that could have been selected throughout the project. All the projects
		presented and listed here will then be compared and the choices
		made will be underpinned in the parts \vref{hardware} and \vref{software}.
		
		\section{Building an electric car}\label{buildingcar}
		
		Several projects have aimed to realize an electric car robot, sometimes autonomous. 
		Having in mind these projects can give some piece of advice on how it is common 
		to design such systems.
		\\\indent Some projects have used the on-board computer \textit{Jetson Nano}. 
		One can name the following robots : \textit{Jetbot} \cite{jetbot}, \textit{Kaya} \cite{kaya}.
		These robots are both designed by \textit{Nvidia}, the designers of the 
		\textit{Jetson} board series\footnote{Jetson Nano, Xavier, TX1, TX2 \dots}. It 
		is notably interesting to take as example their hardware selection since they
		use the same embedded computer the tracker use.
		\\\indent Other \textit{RC-cars}\footnote{\textit{Remote Control Car}.} use other types of embedded computers.
		However, the hardware architecture with regard to \textit{RC-cars} is 
		almost always the same. With either the \textit{DeepRacer} of \textit{Amazon} \cite{deepracer}, or
		the \textit{Pi-Car} \textit{raspberry} powered car \cite{rasp}, or 
		\textit{Ghost} car \cite{ghost}, or the \textit{Roscar} \cite{roscar}, or
		the \textit{F1tenth} car \cite{f1tenth}, or the \textit{Donkey} car \cite{donkey}, the architecture
		follows some fundamental principles.
		\\\indent The selection of the hardware is discussed further in the part \vref{hardware}.
		
		\section{Tracking}\label{statearttracking}
		
		Multiple ways of tracking a target exist, in this project the most
		common one was used : a \textit{visual tracking}. Visual tracking
		is basically based on the processing of images or frames taken by a camera. 
		\\\indent More precisely, the mission of the tracker implemented
		in this project is to follow
		a specific object in a frame
		\footnote{Given by a camera for instance.}, to be able to 
		learn the object in that process, and even to be able to recover the target after 
		any occlusion. Looking at the deep learning models and algorithms 
		which have been developed in the recent years,
		some could be more labeled as detection algorithms, do not really learn the specificities 
		of the target, and even track multiple targets. For instance, the network \textit{Yolo} 
		\cite{bjelonicYolo2018} is rather designed for multi-tracking and 		not to follow a specific target. 
		\\\indent Considering deep learning models that are able to 
		track a specific target the database \textit{GOT-10k} has produced
		an enumeration and ranking of the current most effective techniques \cite{trakinglist}.
		\\\indent For sure, in order to automate the robot, a detection algorithm
		could precede the tracking.
		 
		
		\section{Using ROS}
		
		Developing with \textit{ROS} gives also an incredible
		advantage to any roboticist. In fact, some \textit{ROS packages}
		which are totally open-source already provide cutting-edge algorithms
		such as sensor fusion by Kalman filtering, \textit{SLAM}
		\footnote{Simultaneous Localization And Mapping.}. Unfortunately, 
		applied to specific object tracking few \textit{ROS} packages were developed, and
		the existing ones do not use deep learning techniques at all.
		\\\indent Nevertheless, some \textit{ROS interfaces} were implemented
		for \textit{ROS}, which can allow one to use a tracking algorithm
		compatible with the \textit{ROS interface}, also often named
		\textit{bridge} \cite{mtf}. The main problem is mainly that the implemented
		programs are not up-to-date which renders the integration
		in \textit{ROS} sometimes almost impossible due to software dependencies.
		\\\indent Apart from that, several projects, almost all shared on
		\textit{github}, implement tracking solutions. However, they 
		either do not use deep learning, or are not supported by \textit{ROS} 
		anymore. The book \textit{ROS Robotics Projects} presents
		a well-documented tracking project, though no
		deep learning techniques are used \cite{rosprojects}. Yet, 
		the \textit{ROS} architecture inspired some of the realization 
		of the own architecture of this project.

%%%%%%%%%%%%%%%%%%%%%%%%%%%%%%%%%%%%%%%%%%%%%%%%%%%%%%%%%%%%%%%%%%%%%%%%%%%%%%%%%%%%%%%%%%%%%%		

\chapter{Hardware}\label{hardware}

\introductionLettrine{F}{irst} of all, before even thinking about the tracking process, and
the behavior of the robot, the existing platform had to be fixed, and if needed adapted
to the current needs of the project. This part starts by presenting the latest architecture 
of the tracking robot, an then breaks this architecture apart by giving an overview of
each challenge which had to be tackled.
	
	\section{Hardware architecture}
		
		\subsection{Overview}\label{hardwareoverview}
		
		It is crucial to chose wisely the hardware of the robot in the first place, since
		the software could be seen then as an exploitation of the hardware of 
		the robot. The initial hardware of the robot did comprise
		a servomotor, a \textit{brushed DC motor}\footnote{More is explained 
		on that in the section \vref{motor}.}, and a motor controller, or \textit{ESC}, which
		stands for \textit{ Electronic Speed Controller}.
		\\\indent This initial architecture had to be adapted for several reasons.
		The latest hardware architecture of the tracking robot is
		presented on figure \vref{hardwarearchitecture}, which
		shows each component and how each interacts with each other.
		\\\indent Let's explain a bit what are the crucial parts
		that constitute a tracking robot. The target tracking process used
		in this project is based on image and depth inputs, but to be more
		precise, on a \textit{stereo camera}. A \textit{stereo camera}
		provides basically two types of information : depth and color.
		For this purpose the \textit{ZED camera} of \textit{Stereolabs}
		was used \cite{zeddoc}. The processing of the frames
		given by the camera is done on the embedded computer
		\textit{Jetson Nano}. Regarding the actuators, a servomotor
		controls the bearing of the car, and a motor controls the speed
		of the car through an \textit{ESC}. In order to communicate
		with the motor controller, or \textit{ESC} a \textit{PWM shield}
		was needed. Basically, a \textit{PWM} signal, which stands for
		\textit{Pulse Width Modulation} in this case, is a certain type
		of signal which are characterized by their duty cycle. This 
		fundamental parameter is equivalent to the command sent 
		to the \textit{ESC}. At this end, the \textit{PWM shield} PCA9685
		of Adafruit was used \cite{adafruitpwm}. Ultimately, since
		in the field of mobile robotics, robots tend to be mobile by definition,
		the tracking robot had to be powered on its own. To 
		meet that constraint two batteries were added. From all that has
		been mentioned in this paragraph, only the initial servomotor was
		saved.
		
		\begin{figure}
			\label{hardwarearchitecture}
		\end{figure}

		\subsection{Components list}
		
	\section{Low speed control}\label{motor}
	
		\subsection{Motor}
		
		After having fixed the existing \textit{RC car}, the speed was unfortunately 
		too high to meet the likely processing speed of the embedded computer.
		The next step
		was then to find a way to slow down the robot. 
		\\\indent The first constraint was to avoid changing the mechanics as much as
		possible. Changing the gear box, without buying a new \textit{ESC} and a new
		motor could have been possible. Yet, it would have required to alter 
		the entire mechanics of the \textit{RC car} platform. Thus, this option was 
		not considered.
		\\\indent How is it possible then not to modify the mechanics and
		at the same time to decrease the speed? If the mechanics was to 
		be kept identical, that is to say that the power chain should 
		remain untouched, the motor has to remain of the same dimensions.
		This is why as new motor, a \textit{brushed DC motor} was in part 
		chosen. Economically talking, a \textit{brushed} motor is far cheaper
		in comparison to its counterpart the \textit{brushless} motor. Yet, the
		speed has still not been decreased.
		\\\indent To understand how it is actually possible to 
		slow down a \textit{DC motor} it is essential to fathom the 
		existing relation between the torque applied to the charge
		and the \textit{RPM}\footnote{Revolutions Per Minutes.},
		or in other words the angular speed 
		of the shaft of the motor. Basically, the angular speed is
		inversely proportional 
		to the torque. Thus, in order to decrease the angular speed, a 
		\textit{brushed DC motor} with more torque had to be chosen. \cite{bonanza, motor}
		
		\subsection{Motor Controller}
		
		A \textit{DC motor} without a controller is not very useful. At this 
		stage, a controller had to be selected in order to set by commands
		the angular speed of the motor.
		Numerous strategies exist to regulate the speed of a \textit{DC motor}.
		For instance, it is possible to use specific motor controllers \cite{f1tenth}, 
		transistors, or \textit{ESCs}. From all the projects introduced
		in the state or the art part \vref{buildingcar}, a majority 
		uses \textit{ESCs}.
		\\\indent Yet, the race car of the \textit{MIT}\footnote{Massachusetts Institute
		of Technology, USA.} employs \textit{VESC}, which stands for \textit{Vedder 
		Electronic Speed Controller} \cite{mitracecar}. Such \textit{ESCs} were designed
		to provide more flexibility in the control of the low speeds. However, their price
		were far too high to be considered in this project.
		\\\indent All in all, a new brushed motor \textit{ESC} was bought.
		\\\indent \textit{ESCs} were first designed for \textit{RC cars}, that is to say for 
		remote control in general. This is why, the commands that should be given to them
		as inputs are signals transmitted by remote controller, \textit{PWM signals}. In order
		to generate such signals the \textit{GPIO}\footnote{Which stands for \textit{General Purpose Inputs Outputs}.}
		pins or outputs of the \textit{Jetson Nano} 
		provide some relevant functionalities. The jetson nano has two channels that
		can generate \textit{PWM} signals. Yet, the \textit{PWM shield}  PCA9685
		of Adafruit is more reliable, since its function is exclusively to generate
		such signals. The selected solution was to use the \textit{GPIO} pins 
		to generate \textit{I2C}\footnote{Two-wire communication 
		protocol that allows to address several devices or give several 
		orders at the same time to external devices from a microcontroller.} signals to communicate
		with the \textit{PWM shield}. The \textit{PWM shield}
		then outputs the right \textit{PWM signals}.\cite{tk1servo,nanogpio,nanogpiolayout} 
	
	\section{Power consumption}
		
		\subsection{How many?}
		% How many batteries?
		The first question that should came to mind is not which type of battery
		should be bought, or even
		how much power the hardware requires, but how many batteries the robot 
		needs. It is common in robotic applications and more 
		extensively in any system to split the power supply into two parts.\cite{racecarj} 
		\\\indent Actuators, that is to say for instance motors
		or servomotors, tend to require far more power than the embedded computer. Besides, 
		should the actuators draw too much current\footnote{It often happens that the 
		actuators may induce current peaks.}, the embedded computer will switch off. Thus, 
		as a matter of safety, it is crucial to isolate the power source of the actuators
		from the one of the embedded computer. This explains the choice of two separate 
		batteries.
		% split the power source in more than two parts : usb hub --> battery for the computer + for the actuators : minimum.
		\\\indent By and large, the takeway is that the embedded computer must not, 
		in any circumstances, be switched off, since the hardware will still 
		apply the last commands stored in the more detrimental cases. It is 
		also highly recommended to use a self-powered \textit{USB-Hub}
		to plug any subsequent device to the embedded computer, which 
		may draw too much current from the embedded computer \cite{racecarj}.
		That advice was not applied in this project, and could be
		seen as a point of improvement.
		\\\indent All these choices were further underpinned by 
		the projects presented in \vref{buildingcar}.
		
		\subsection{Which ones?}
		% Battery for the ESC and motor : actuator.
		On the one hand, regarding the motor, which is the actuator that 
		draws the more power among the hardware of the robot, an independent 
		\textit{Li-Po} battery has been chosen. This choice
		was further promoted, for \textit{Li-Po} batteries, in 
		comparison to \textit{NiMH} and \textit{Ni-Cd} batteries, 
		tend to be more efficient and start to gradually 
		supersede them.
		% How do we power the computer, and the motor? Pb when powered via micro-usb ---> powered via barrel jacket :
		% battery for the servomotor
		\\\indent On the other hand, regarding the power source of the embedded computer, with 
		the normal power input through \textit{micro-usb} the power supplied
		was maximum 10 Watts, and the jetson nano used to be turned off. As a matter of fact,
		after some tests, it appears that it was due to an under-powering of the 
		\textit{Jetson Nano}, although the \textit{Jetson Nano} was not connected
		to any other device. Multiple ways exist to power the \textit{Jetson Nano}.
		Powering the board through the \textit{Power Jack}, which gives maximum 20 Watts, 
		was the chosen solution. Hence, the battery \textit{Krisdonia 25000mAh} was bought to meet that 
		requirement \cite{nanopowerbank}. In addition, it enables to plug more external devices to the \textit{Jetson Nano}
		and fasten its overall performances.
		\\\indent The servomotor is still not powered. Servomotors are commanded
		via \textit{PWM signals}. Those ones are generated with the \textit{PWM shield}, also 
		used for the communication with the \textit{ESC}. The \textit{PWM shield} needs to power the 
		servomotor though. This is done by using the \textit{V+} external power input of
		the \textit{Adafruit shield}. Since the \textit{Krisdonia 25000mAh} is powerfull 
		enough with a \textit{DC ouput} for the\textit{Jetson Nano} and two other independent
		\textit{USB power outputs} with a $5V/3A$, the servomotor is powered through the \textit{PWM shield}
		with the \textit{Krisdonia 25000mAh} battery.
		% Present the battery adviced by jetson hacks.
		% How many power do our hardware need?
		\cite{jetsonhacksmorepower,nanoguide,elinuxdoc}
		
	\section{Remote control and processing}
		
		% Remote control of the robot, supervising, displaying information
		% other interest : using more powerful GPUs for deep learning applications : processing the frames on 
		% a remote computer.
		When the tracking robot is in mission, it is essential that 
		at each time the information gathered and processed by the
		robot can be displayed and monitored by a human agent.
		For instance, it is useful to store the data on a remote 
		computer, to display some curves, or 
		simply to command the robot remotely when the tracker encounters
		some hurdles. Communication can be established via \textit{SSH}\footnote{Secure Shell.}
		in a terminal environment. Yet, the robot needs to be able 
		to be connected to a network, and unfortunately, the 
		\textit{Jetson Nano} does not provide any way to do that. The most common way 
		to achieve that is by using a \textit{WiFi Module}. A huge 
		variety of such devices can be find on the market, the question 
		is then which one of those is the most suitable for the 
		targeted application.
		\\\indent Another aspect of the decision should take into 
		account that it would be more efficient and sustainable 
		to be able to process the frames of the camera on
		a remote computer more powerful than the \textit{Jetston Nano}.
		However, to send frames via \textit{WiFi} the \textit{ WiFi Module}
		must possess a relevant bandwidth and a faster than average communication
		speed. For this reason the module \textit{8265NGW} of Intel was chosen, which 
		delivers up to 867Mbps \cite{wifi}.
		
	\section{Performances of the embedded computer}
	
		The performances of the embedded computer are crucial in 
		any robotic application.
		At first the \textit{Jetson Nano} was too slow 
		to do anything else than acquiring the image of the \textit{ZED camera}.
		In order to increase those performances a bigger \textit{SD card} of 
		about 64 GB had to be added.
		\\\indent Improving the processing speed of the \textit{Jetson Nano}
		could also be done by artificially increasing the \textit{RAM}\footnote{Random Access
		Memory.} of the board, which comes with a limited 4 GB of \textit{RAM}. The latter
		was not put in place, it is highly recommended in the future though. \cite{swap}
	
	\section{Unit Tests}

		Before trying to integrate anything, that is to say 
		to use the hardware inside a \textit{ROS} framework 
		or to merge hardware and tracking, the hardware was
		tested independently.
		\\\indent Especially, regarding the actuators, 
		being the servomotor and motor, a low-level library 
		was written in python, \textit{hardware.py}, which is
		presented in the appendix \vref{hardwarelib}. 
		It belongs to the software of course, but since the low-level 
		library is aimed to directly drive the hardware, this 
		section can be considered inside the hardware part. 
		\\\indent In fact, this library enables to create
		a \textit{python object} for each piece
		of hardware. Each object provides
		some high level methods, which can 
		then be used in a more complex \textit{ROS}
		architecture. Parameters were also set 
		in order to transfer a low-level command
		to a high-level command.
		

\chapter{Software}\label{software}

	\introductionLettrine{T}{he} hardware selection has been discussed in the part \vref{hardware}. 
	Once the hardware architecture is defined, and tested. The software 
	part must be implemented in order then to merge them together. The software
	comprises mainly two challenges : the realization of the tracking and the
	conception of the \textit{ROS} architecture.
	
	\section{Tracking}\label{tracking}
	
		% NOTES
		% strategy of development, position of the problem
			% pretrained model
			% then complicate it
		% Why GOTURN? opencv, easy of use first, and then maybe my own, can be trained
			% add the goturn paper
		% ROS interfacing and GOTURN
		% Unit Tests : really good for human faces (it was apparently more trained on human faces), 10-30FPS with 
		% the webcam of the computer, a static image
		
		\subsection{Overview}
		
		As explained in \vref{hardwareoverview} and \vref{statearttracking}, in this project
		the tracking algorithm is implemented with a deep neural network and takes as input
		the color frames of the \textit{ZED camera}. The goal is
		to have a tracking algorithm able to follow a specific target in 
		a sequence of frames.
		\\\indent The approach to the problem was to find a tracking algorithm or model 
		which is pretrained. Basically, starting with the easiest solution and 
		then, if needed, refining it. Another idea was also to 
		have the model running in a controlled environment and then, once tested, 
		interfacing it with \textit{ROS}.
		
		\subsection{Goturn}
		
		Among the numerous and variegated tracking solutions presented in 
		\cite{trakinglist} the pretrained model \textit{GOTURN} was chosen.
		\\\indent First of all, only the tracking algorithms that were
		implemented in \textit{python} were selected for compatibilities issues.
		\textit{GOTURN} was also the best choice regarding the ease of 
		programming. It is implemented inside the \textit{OpenCV}
		library, which is one of the most widely used
		computer-vision library worldwide, and which renders it far easier
		to integrate the tracking model in the \textit{ROS} architecture
		afterward. \cite{goturn}
		\\\indent \textit{GOTURN} is also sustainable and flexible, for it is
		possible to acquire the untrained model for a more 
		specific application.\cite{goturnpy}
		
		\subsection{ROS interface}
		
		Making the \textit{GOTURN} tracker work inside
		the \textit{ROS} framework was not that easy, 
		although it was easier than the other available solutions.
		% python environment for opencv, docker
		Running the \textit{GOTURN} function provided 
		in the \textit{OpenCV} library needed 
		to have a version of \textit{OpenCV} higher than $3.4.2$. However, since
		\textit{ROS} relies on \textit{Python 2.7}, 
		it only comes with an older version. I was not
		able to uninstall this library since it was
		a dependencies for other \textit{ROS} packages. The 
		workaround was to create a \textit{Python Environment}
		using \textit{virtualenv} on \textit{Ubuntu}, and then 
		to specify this \textit{Python} interpreter of this
		particular environment in the code of the tracker
		using the \textit{shebang}, the first line of code
		where the interpreter is specified.
		% ROS interface, image message

		\subsection{Unit Tests}
		
	\section{ROS}\label{ros}
	
		\begin{comment}
		
		The Bascis of ROS
			master
			topic 
			nodes
			services
		Overview of the ROS architecture
			Position of the problem
			How does the ROS architecture work? Function of each node
			Two version of the ROS architecture
				remote processing
				local processing
		Speed regulation
			PID controller
		Bearing regulation
			PID controller
		Unit tests without the hardware : everything was simulated
			Only the remote one was tested
				
		\end{comment}
		
		\subsection{ROS Basics}
		
		\subsection{Overview}
		
		\subsection{Speed regulation}
		
		\subsection{Bearing regulation}
		
		\subsection{Unit Tests}
			
\chapter{Results : integration}\label{results}
	
	\introductionLettrine{T}{his} part gives an analysis of the latest results
	obtained in this project. Basically, the part tackles
	the integration of the software in the hardware.
	
	\begin{comment}
	
		Results of the integration of the hardware with the software:
		test with the remote architecture and the Ethernet cable
			5 FPS in average
			curve of the error
			several pictures showing that the robot follows the target
	
	\end{comment}

\chapter{Setup}\label{setup}

	\introductionLettrine{T}{his} part presents the material needed and the 
	necessary conditions to replicate the \hyperref[results]{results}\footnote{cf. \vref{results}.} of
	this project. Should you need some explanations on the project, please
	refer to parts \vref{context}, \vref{stateofart}, \vref{hardware}, and \vref{software}.

	\section{Hardware setup}

	% TODO
	% wiring, blueprint of the wiring (or maybe use the same than the hardware architecture)
	
	\begin{itemize}
		\item [\ding{55}] \underline{Computers} : Jetson Nano, Ubuntu 18.04, remote computer Ubuntu 18.04. 
	\end{itemize}
	
	
	\section{Software setup}