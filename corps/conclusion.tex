\clearpage

\section*{Conclusion}

\begin{comment}
	
	Compare to the paper on the person following which had a 20 Fps. (check whether or not they did it in c++)
	What has been implemented and what has not been implemented.
 E	Ideas for further development : ROS2, c++

\end{comment}

\introductionLettrine{T}{he} tracker robot was able to track 
a specified target, human targets were followed 
more accurately. The \textit{ROS} architecture
that has been implemented is flexible and could
be used for further improvements.
\\\indent Regarding the performances, in 
comparison to a person following robot \cite{personfollowing}
tracking at a rate of 20 \textit{FPS}, the tracker
implemented in this project tracks at a maximum of 6 \textit{FPS}.
The performances could be improved though. The person 
following robot was implemented in \textit{C++} in 
comparison to the tracker in this project, which was 
implemented entirely in \textit{Python}.
\\\indent It is also important to mention that not 
all the components ordered, and presented on 
figure \vref{hardwarepng}, had arrived before
the end of the project. Having the best components would 
have certainly improved the performances
of the tracking robot. Especially, the tracker 
would need to be tested with the \textit{Krisdonia 25000mAh}
power bank and the \textit{Intel 8265 Wifi module}.
\\\indent Another point of refinement could be the 
training of the \textit{GOTURN} model to make 
it more accurate, the realization of a personal 
deep learning network, or the test of another one.
\\\indent For any further implementation, 
and especially if realizing a personal 
deep learning model is the goal, switching to \textit{ROS2}
could be a wise move, since \textit{ROS2}
is coded in \textit{Python 3}, and would make it easier to 
use the common deep learning libraries out of the box, such 
as \textit{PyTorch} or \textit{Tensorflow}. 
For any deep learning implementation with the \textit{Jetson Nano}, 
it is recommended to read through the \textit{Nvidia} tutorials  first\cite{dptuto}.
\\\indent To sum up, the latest tested prototype, as presented in the \href{https://github.com/klipfel/tracker-v1}{github},
did not implement the speed control. The idea was to obtain first acceptable performances
for the tracking algorithm, which is the juggernaut of the robot. Several 
things should be tested for any motivation to continue this project:
\begin{itemize}
	\item[\textbullet] Transpose the \textit{ROS} architecture in \textit{ROS2}: it would then 
	be easier for any \textit{Python} support.
	\item[\textbullet] Implementing in \textit{C++}: the performances of
	\textit{C++} compared to those of \textit{Python} are unequivocal. The \textit{Jetson Nano}
	was designed to be more suitable for \textit{C++} programming. It would be more
	flexible for \textit{GPU} acceleration.
	\item[\textbullet] Hard-code the tracking model in \textit{PyTorch, tensorflow}, so that 
	it would be possible to use \textit{GPU acceleration} on the \textit{Jetson Nano}, cf. 
	integration presented in \vref{test4}.
	\item[\textbullet] Test the \textit{Wifi} module and the \textit{Krisdonia 25000mAh}
	of the \textit{ROS} architecture 
	presented in \vref{hardwarepng}, when delivered.
	\item[\textbullet] Test the speed control as presented in figure \vref{rosarchitecture} and in \vref{speed}.
\end{itemize}