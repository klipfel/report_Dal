\clearpage

\section*{Conclusion}

\begin{comment}
	
	Compare to the paper on the person following which had a 20 Fps. (check whether or not they did it in c++)
	What has been implemented and what has not been implemented.
	Ideas for further development : ROS2, c++

\end{comment}

\introductionLettrine{A}{t} the end, the tracker robot was able to track 
a target, although human targets were followed 
more accurately. The \textit{ROS} architecture
that has been implemented is flexible and could
be used for further improvements.
\\\indent Regarding the performances, in 
comparison to a person following robot \cite{personfollowing}
tracking at a rate of 20 \textit{FPS}, the tracker
implemented in this project tracks at a maximum of 6 \textit{FPS}.
The performances could be improved though. The person 
following robot was implemented in \textit{C++} in 
comparison to the tracker in this project, which was 
implemented entirely in \textit{Python}.
\\\indent Another point of refinement could be the 
training of the \textit{GOTURN} model to make 
it more accurate, the realization of a personal 
deep learning network, or the test of another one.
\\\indent For any further implementation, 
and especially if realizing a personal 
deep learning model is the goal, switching to \textit{ROS2}
could be a wise move, since \textit{ROS2}
is coded in \textit{Python 3}, and would make it easier to 
use the common deep learning libraries out of the box, such 
as \textit{PyTorch} or \textit{Tensorflow}. 
For any deep learning implementation with the \textit{Jetson Nano}, 
it is recommended to read through the \textit{Nvidia} tutorials  first\cite{dptuto}.